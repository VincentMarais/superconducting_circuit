\mydef{Lagrangian (L)}{
	\[ 
	L = T - V 
	\]
	where \(T\) is the kinetic energy and \(V\) is the potential energy.
}

\mydef{Hamiltonian (H)}{
	\[ 
	H = \dot{q}_i p_i - L 
	\]
	where \(\dot{q}_i\) are the generalized velocities and \(p_i\) are the canonical momenta.
}

\mydef{Euler-Lagrange Equations}{
	\[ 
	\frac{d}{dt} \left( \frac{\partial L}{\partial \dot{q}_i} \right) - \frac{\partial L}{\partial q_i} = 0 
	\]
}

\mydef{Canonical Momenta (p\_i)}{
	\[ 
	p_i = \frac{\partial L}{\partial \dot{q}_i} 
	\]
}

\mydef{Hamilton’s Equations of Motion}{
	\[ 
	\dot{q}_i = \frac{\partial H}{\partial p_i}, \quad \dot{p}_i = -\frac{\partial H}{\partial q_i} 
	\]
}

\mydef{Phase Space}{
	The space of canonical variables \((q, p)\).
}

\mydef{Cyclic Coordinates}{
	Coordinates that do not appear in the Lagrangian, leading to conserved canonical momenta.
}

\mydef{D’Alembert’s Principle}{
	\[ 
	\sum ( \dot{p}_i - F_i) \cdot \delta r_i = 0 
	\]
}

\mydef{Virtual Displacement (\(\delta r_i\))}{
	An infinitesimal displacement consistent with the constraints, carried out at a fixed time.
}

\section*{Important Theorems}

\myth{Hamilton’s Principle}{
	The motion of a system extremizes the action \(S\):
	\[ 
	\delta S = \delta \int_{t_1}^{t_2} L \, dt = 0 
	\]
	This leads to the Euler-Lagrange equations.
}

\myth{Noether’s Theorem}{
	Every differentiable symmetry of the action corresponds to a conservation law.
}

\myth{Liouville’s Theorem}{
	The phase space distribution function is conserved along the trajectories of the system.
}

\section*{Key Concepts}

\mydef{Newtonian Mechanics}{
	Describes the dynamics of particles using vector spatial coordinates or generalized coordinates.
}

\mydef{Lagrangian Mechanics}{
	Focuses on the Lagrangian function and provides a scalar approach to dynamics, making it easier to handle different coordinate systems and constraints.
}

\mydef{Hamiltonian Mechanics}{
	Introduces canonical momenta and phase space, providing a different formalism that is especially useful in statistical mechanics and quantum mechanics.}
