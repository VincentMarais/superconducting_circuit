
The superconducting state is a phase of matter or, more precisely, 
a second-order phase transition of matter at a temperature $T_c$ that 
induces different properties. The historical property is the low resistance 
($R < 10^{-5} \Omega$) discovered by Heike Kamerlingh Onnes.


\fig{1}{part/intro/Mercury_superconducting_transition.pdf}{Mercury
superconducting
transition}

\newpage
In 1933, in Berlin, 
Walther Meissner and Robert 
Ochsenfeld showed that the magnetic 
field $B$ is “expelled” from superconductors. 
This means that when subjected to an external magnetic 
field, superconductors divert the field lines so that the 
magnetic field vanishes inside. The superconducting material 
behaves as a perfect diamagnet \cite{mangin_superconductivity_2017} p.20.

\fig{0.5}{part/intro/EfektMeisnera.pdf}{Diagram of the  Meissner-Ochsenfeld effect. 
Magnetic field lines $\mathbf{B}$, represented as arrows,
 are excluded from a superconductor when it is below its critical temperature $T_c$ \cite{noauthor_meissner_2024}.}



